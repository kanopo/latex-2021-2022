\documentclass{article}
\usepackage{graphicx}
\usepackage{float}
\usepackage{amsmath}
\usepackage{amsfonts}
\usepackage{amssymb}
\usepackage{hyperref}
\usepackage{esint}
\usepackage[utf8]{inputenc}
\usepackage[a4paper, portrait, margin=0.75in]{geometry}
\setlength\parindent{0pt}
\usepackage[italian]{babel}
\usepackage{blindtext}




\hypersetup{
    colorlinks=true,
    linkcolor=black,
    filecolor=magenta,
    urlcolor=blue,
    pdftitle={Tecnologie internet},
    pdfpagemode=FullScreen,
}


\begin{document}
    \author{CCEnigma}
    \title{Riunione soci}
    \date{30 Gennaio 2022}

    \maketitle
    \tableofcontents

    \listoffigures
    \listoftables



    \section{CIAC}
    Spiegazione di cosa è ciac e spiegazione sull'utilizzo della sala convegni(rocca).

    \section{Giornata della memoria}
    5 Febbraio in parrocchia

    \section{Festival}
    Entrare il sala nostra per agevolazioni.

    Data ipotetiche giugno/settembre, fissare incontro con sala nostra per andare avanti.

    \section{Giornata contanti}
    21 Giugno.


    \section{Apertura}
    Apertura il 4 di Febbraio.
    Non sembrano esserci contari, persone minime da 3.


    \section{Mostre}
    \begin{itemize}
        \item ci sarebbe da alzare i pannelli
        \item stocco ha una quinta del toschi per mostre future.
    \end{itemize}


    \section{Podcast}
    Riunione da fissare per la settimana del 7.


    \section{Idee eventini}
    \begin{itemize}
        \item incontri sui referendum(contattare qualuno per parlare della cannabis ed eutanasia)
        \item Febbraio parlamentare turca scappata in Germania(lei arriva il 5, la manifestazione il 12)
    \end{itemize}


    
    \section{Pulizia}
    C'è da lasciare pulit oogni volta che si esce.
\end{document}
