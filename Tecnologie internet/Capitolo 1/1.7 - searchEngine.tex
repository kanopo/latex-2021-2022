\subsection{Search Engine for the Web}
\subsubsection{Web search}
L'operazione di ricerca nel web è molto simile al ricevere informazioni specifiche da un determinato server, solo che utilizza il linguggio naturale per cercare i documenti html.
\subsubsection{Motori di ricerca}
I motori di ricerca hanno tre caratteristiche principali:
\begin{enumerate}
    \item Crawling:
        navigare in internet per cercare tutti i contenuti
    \item Indexing
        salvare e organizzare tutti i documenti trovati durante il Crawling
    \item Ranking
        riorganizzare i risultati della ricerca in base all'accuratezza nel rispondere al'argomento ricercato(se cerca cane, non mi aspetto di trovare navi militari ecc)
\end{enumerate}

L'indice di tutti i documenti deve essere regolarmente aggiornato per includere nuove pagine.

\subsubsection{Web crawling}
Un crowler basilare(robot, bot, spider) consiste in:
\begin{itemize}
    \item Una coda di URI da visutare
    \item Un metodo per ricevere le risorse delle pagine con HTTP
    \item Un page parser per estrarre informazione
    \item Una connessione all'indicizzatore del motore di ricerca
\end{itemize}
Il procedimento ordinario consiste in: prendere un URL, analizzare e trovare URL all'interno del testo per aggiungere questi link nella coda di analisi e continuare a indicizzare le pagine.

\subsubsection{PageRank}
Algoritmo per il page ranking inventato da Larry Page, utilizzato da Google per il suo motore di ricerca.
PageRank è un algoritmo che analizza i link fra le risorse e il risultato viene utilizzato per il ranking.
Avendo una pagina $p$ e questa pagina è collegata alle pagine $q_1, ..., q_n$.
\begin{displaymath}
    PR(p) = (1-d) + d[\frac{PR(q_1)}{C(q_1)}+ ... + \frac{PR(q_n)}{C(q_n)} ]
\end{displaymath}