\subsubsection{Client-side Storage}
Le applicazioni web possono usare le API del browser per salvare dati nel client,
\begin{itemize}
    \item Web Storage
    \item Cookies
    \item File system access
\end{itemize}

\subsubsection{Scripted Graphics}
Utilizzo L'HTMl canvas in accoppiata con JS per disegnare lato client, rendendo meno opesante l'app lato server.

\subsubsection{TypeScript}
TypeScript implementa un controllo sulle variabili, JS non controlla se una stringa è davvero una stringa, TS lo fa!!(Come tutti i linguaggi normaliXD)
TS offre le generics come in C++ dentro a JS.

\subsubsection{Frameworks and Development Tools}
I framework vengono usati come scheletri per JS permettendo allo sviluppatore di preoccuparsi meno sulla struttura del codice e la manutenzione.
Vantaggi di Framework in JS:
\begin{itemize}
    \item Efficenza
    \item Sicurezza
    \item Costo
\end{itemize}

\subsubsection{Frameworks}
\begin{itemize}
    \item \href{https://angular.io/}{Angular}
    \item \href{https://reactjs.org/}{React}
    \item \href{https://v3.vuejs.org/}{Vue}
\end{itemize}

\subsubsection{Altri strumenti}
\begin{itemize}
    \item Mobile-first sites development tools: Bootstrap
    \item Documentation tools: Swagger, JSDog, JGrouseDog, YUIDoc, Docco
    \item Testing tools: Jasmine, Mocha, PhantomJS, Protractor
    \item Debugging tools: JavaScript Debugger, Chrome Dev Tools, ng-inspector, Augury
    \item Security tools: Snyk, Node Security Project, RetireJS, Gemnasium, OSSindex
    \item Code optimization and analysis: JSLint, JSHint, ESLint, Flow
\end{itemize}