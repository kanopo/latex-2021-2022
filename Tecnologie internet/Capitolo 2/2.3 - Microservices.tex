\subsection{Microservizi}
Un Microservizio è un servizio indipendente che assolve us una certa problematica specifica, (sempre filosofia UNIX), si tende ad avere un applicazione che necessita di tanti microservizi che si occupano di spefifiche azioni.
Vengono trattati come delle scatole nere che adempiono ad un compito!

\subsubsection{Stili di comunicazione}
\begin{itemize}
    \item Sincrono bloccante
    \item asincrono non bloccante
    \item richiesta risposta
    \item event-driven
    \item dati comuni
\end{itemize}

\subsubsection{Implementing Microservice Communication}
Good practices:
\begin{itemize}
    \item backwards compatibility
    \item interfaccia esplicita
    \item API indipendente dalla tecnologia
    \item servizi semplici per il cliente
    \item nascondi le implementazioni interne
\end{itemize}

Tecnologie:
\begin{itemize}
    \item Remote Procedural Calls
        \begin{itemize}
            \item SOAP
            \item gRPC
            \item CORBA
        \end{itemize}
    \item REST
    \item GraphQL
    \item Message Brokers
\end{itemize}