\subsection{RESTful Services}
Rest sta per Representational State Transfer, si appoggia su comunicazioni stateless client-server.

\subsubsection{Componenti architetturali di REST}
\begin{itemize}
    \item Risorsa: Sono la chiave di un vero design RESTful. Le risorse sono identificate da richieste, le risorse sono concettualemente separate dalla rappresentazione che viene inviata al client.
    \item Risorsa Web: Risorse piccole (Tipo la filosofia UNIX o KISS), se dentro alla risorsa ci sono elementi interessanti che però sono in altre risorse, è possibile includerle mediante link.
    \item Non c'è uno stato di connessione
    \item La risposta dovrebbe essere cacheable
    \item Possono essere usati dei server proxy
\end{itemize}

\subsubsection{Servizi RESTful}
\begin{itemize}
    \item indipendenti dalla piattaforma
    \item indipendenti dal linguaggio
    \item basati su standard
\end{itemize}

\subsubsection{Routes vs. Endpoints}
Gli endpoint sono accessibili dalla API, una route è il "nome" usato per accedere all'endpoint mediante URL.
Esempio: http://example.com/wp-json/wp/v2/posts/123
\begin{itemize}
    \item la route è wp/v2/posts/123
    \item La rout ha 3 endpoint
\end{itemize}

\subsubsection{Guide di design REST}
\begin{enumerate}
    \item Non usare indirizzi URL fisici: prova.com/ciao.xml, ma usare indirizzi logici come prova.com/saluti/002
    \item Le query devono restituire il minor numero di dati possibile, se la query restituisce tanti oggetti sarebbe meglio restituirne 10 alla volta
    \item La risposta di un servizio REST può essere qualsiasi cosa, è necessaria una documentazione chiara.
    \item Se una risposta ha funzionalità aggiuntive, la query dovrebbe comunicare gli URL di possibili azioni e non lasciar creare URl all'utente.
\end{enumerate}

\subsubsection{Consumare un servizio REStful con JQUERY}
Il servizio è: http://rest-service.guides.spring.io/greeting.


Il servizio risponde: {"id":1,"content":"Hello, World!"}.

\begin{lstlisting}
public/hello.js
$(document).ready(function() {
    $.ajax({
        url: "http://rest-service.guides.spring.io/greeting"
    }).then(function(data) {
        $('.greeting-id').append(data.id);
       $('.greeting-content').append(data.content);
    });
});

<!DOCTYPE html>
<html>
    <head>
        <title>Hello jQuery</title>
        <script src="https://ajax.googleapis.com/ajax/libs/
        jquery/3.6.0/jquery.min.js"></script>
        <script src="hello.js"></script>
    </head>
    <body>
        <div>
            <p class="greeting-id">The ID is </p>
            <p class="greeting-content">The content is </p>
        </div>
    </body>
</html>
\end{lstlisting}

\subsubsection{Express}
Express.js permette di creare velocemente e facilmente API robuste.