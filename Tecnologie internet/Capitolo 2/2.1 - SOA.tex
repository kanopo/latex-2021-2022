\section{Architettura Orientata ai Servizi}
SOA è l'Architettura prevalente per distribuire informazioni.
SOA mette a disposizione l'interfaccia per un servizio(procollo, formato, comportamento) e chiunque ha l'accesso al servizio, rispettando i criteri, può effettuare richieste.
\subsection{Quality of Service(QoS)}
Un astessa interfaccia può corrispondere a diverse implementazioni, QoS è legato ad aspetti non-funzionali che influenzano come il servizio viene consumato:
\begin{itemize}
    \item Performance
    \item Availability
    \item Robustness
    \item Required authorizetions
    \item Cost
\end{itemize}

Il fornitore del servizio ed il suo consumatore devono stabilire un SLA(Service Level Agreement) per gestire le possibilità di utilizzo delle "API".

\subsubsection{SoapUI}
\href{https://www.soapui.org/}{SoapUI} è usato per testare i servizi:
\begin{itemize}
    \item Soap
    \item REST
    \item WSDL
\end{itemize}
