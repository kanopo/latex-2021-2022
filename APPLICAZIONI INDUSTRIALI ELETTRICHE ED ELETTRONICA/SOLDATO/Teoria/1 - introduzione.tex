\section{Introduzione}

Le grandezze principali sono la \textbf{tensione}(campo elettrico) e la \textbf{corrente}(campo magnetico).

\subsection{Tensione}

\begin{figure}[H]
    \centering
    \includegraphics[width=0.7\linewidth]{imgs/tensione}
    \caption{tensione}
    \label{fig:tensione}
\end{figure}

La tensione V è:
\begin{equation}
    \vec{E} = - grad V = -\frac{\delta V}{\delta x} \hat{x}
    -\frac{\delta V}{\delta y} \hat{y}
    -\frac{\delta V}{\delta z} \hat{z}
\end{equation}

Detto anche potenziale elettrico, è l'energia potenziale elettrica normalizzata per la carica.

La tensione è la differenza di potenziale elettrico(d.d.p.).
\subsubsection{Unità di misura della tensione}
La tensione si misura in \textbf{volt} $[V]$.

\begin{itemize}
    \item $[q] = C$ "coulomb" $= A\cdot s$ "Ampere per secondo"
    \item $[E] = \frac{N}{C} = \frac{N}{A\cdot s} =
    \frac{Kg \cdot \frac{m}{s^2}}{A\cdot s}$
    \item $V = \frac{N\cdot m}{A \cdot s}$
\end{itemize}

Ricorda che la tensione dal punto B ad A si chiama $V_{AB}$ e che $V_{AB} = - V_{BA}$.








