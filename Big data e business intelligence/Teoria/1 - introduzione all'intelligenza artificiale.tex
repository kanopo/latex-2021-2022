\section{Introduzione all'intelligenza artificiale}

Gli obiettivi di questo corso:
\begin{itemize}
    \item Capire i dati
    \item Visualizzazione de dati
    \item rappresentazione dei dati
    \item Estrazione della conoscenza
    \item Ingegnerizzazione dei dati
    \item Analsi predittiva
\end{itemize}

\subsection{Introduzione ai Big Data}
I big data sono quei dati che presentano una o più delle seguenti proprietà:
\begin{itemize}
    \item \textbf{Volume}: grandi quantità di dati
    \item \textbf{Varietà}: eterogeneità dei dati per tipologia, struttura, fonte, ecc
    \item \textbf{Velocità}: Dati che vengono generati con grande velocità
    \item \textbf{Veridicità}: dati sconnessi, mancanti, inconsistenti
\end{itemize}

Possoiamo categorizzare i dati in base alla loro struttura:
\begin{itemize}
    \item Dati strutturati
    \item dati non strutturati
    \item dati semi-strutturati
\end{itemize}

\subsubsection{Dati strutturati}
I dati strutturati sono quei dati rappresentabili in tabella, DB, CSV.
\subsubsection{Dati non strutturati}
Non rappresentabili in tabelle.
ES:
PDF, video, audio, immagini, ecc

\subsubsection{Dati semi-strutturati}
Dipendentemente dalla task che si vuole svolgere, si possono trattare i dati come
strutturati, semi-strutturati o non strutturati.


\subsubsection{Qualità dei dati}
La qualità dei dati dipende da:
\begin{itemize}
    \item \textbf{Completezza}
    \item \textbf{Consistenza}
    \item \textbf{Accuratezza}
    \item \textbf{Assenza di duplicati}
    \item \textbf{Integrità}
\end{itemize}

\subsubsection{Fonti dei dati}
Dati interni:
\begin{itemize}
    \item produzione
    \item tranazioni o statistiche di utilizzo
    \item vendite
    \item human resource manager(HR)
    \item customer relationship manager(CRM)
\end{itemize}

Dati esterni sono i dati ottenibili pubblicamente, dati di forum, social network, ecc.

\subsubsection{Dati operazionali}
In una organizzazione ci sono tre tipologie di individui interessati ai dati:
\begin{itemize}
    \item \textbf{Manager}: deve prendere decisioni
    \item \textbf{Operation staff}: motore di ogni singola tasks e fornisce dati interni
    \item \textbf{Data scientist}: responsabile di gestire, analizzare e preparare i dati per il manager e lo staff operativo
\end{itemize}

Visto che questi dati sono tantissimi, si usano Enterprise Resource Planning(\textbf{ERP}) software.

Solitamente i DB utilizzati in questi sistemi, sono disegnati per essere \textbf{OLTP}
(On Line Transaction Processing).
Sono incentrati sulla lettura de dati e non sulla loro analisi.

\subsubsection{I dati operazionali sono un po' meh per le analisi}
Di solito si usano i \textbf{Data Werehouse}, dati fatti su misura che contengono dati consistenti e interessanti per l'analisi.


I Data Werehouse sono il punto di partenza della \textbf{Business intelligence}

\subsubsection{Business Intelligence}
Sistema di modelli, metodi, processi, persone e strumenti che rendono possibile la raccolta regolare ed organizata del patrimonio
di dati generato da una organizzazione.
Attraverso elaborazioni, analisi o aggregazioni, permette:
\begin{itemize}
    \item Trasformazione di informazioni
    \item Conservazione
    \item Reperibilità
    \item Presentazione in forma semplice, flessibile ed efficiente
\end{itemize}

\subsubsection{Ragioni per il fenomeno dei big data}
\begin{itemize}
    \item Computer avanzati
    \item DB non relazionali più performanti
    \item Progresso con il Machine Learning
    \item Progetti open-source
\end{itemize}
\subsubsection{Principali software open-source}
\begin{itemize}
    \item Hadoop
    \item Spark
    \item Python
\end{itemize}

\subsubsection{ML per l'estrazione del'informazione}
ML è un campo di studi dove non si programma per svolgere una task specifica, un 
sistema di ML si \textbf{"allena"} sulla task da eseguire.
Il sistema di ML crea esperienza sulla base degli esempi,
trova patter regolari fra i vari casi e in base all'esperienza acquisita
può prendere decisioni per il futuro.

\subsubsection{Chi è il Data Scientist}
Persona che ha il compito di analizzare i dati con tecniche avanzate,
solitamente con abbastanza conoscenza del dominio e buone soft-skills.

Forte background in:
\begin{itemize}
    \item Computer Science
    \begin{itemize}
        \item DB e SQL
        \item Hadoop
        \item Spark
        \item NoSQL
        \item Python, java, ecc
        \item Artificial Intelligence
    \end{itemize}
    \item Machine Learning
    \item Statistica
\end{itemize}



