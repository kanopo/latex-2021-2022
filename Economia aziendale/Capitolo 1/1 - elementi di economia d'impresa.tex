\chapter{Elementi di economia d'impresa}
L'economia si divide in due:
\begin{itemize}
    \item \textbf{economia aziendale}: studia la differenza fra aziende e il loro funzionamento interno
    \item \textbf{economia politica}: studia i problemi di equilibrio del sistema politico
\end{itemize}

\section{Economia aziendale}
\begin{quotation}
    ``Disciplina che studia le condizioni di esistenza e le manifestazioni di vita dell'azienda.''

    G. Zappa
\end{quotation}

Incentrata sull'azienda vista come ordine economico di un istituto sociale:
\begin{itemize}
    \item famiglia
    \item impresa
    \item ente pubblico
    \item istituto no profit
\end{itemize}

Ne illustra diversi problemi connessi al funzionamento, con l'obbiettivo di disporre di una base conoscitiva
per guidare le decisioni.

Studia in maniera integrata:
\begin{itemize}
	\item l'organizzazione aziendale: studio dei fattori interni all'azienda e dei rapporti tra azienda
		e ambiente.
	\item la gestione aziendale: operazioni interne ed esterne che creano reddito e variano il capitale.
	\item rilevazione di fatti economici: inventari, preventivi, registrazione di gestione, bilancio di esercizio,
		ecc
\end{itemize}

\section{Definizione di azienda}
deriva dallo spagnolo che deriva dal latino "facienda"(le cose da farsi).

\section{Definizione giuridica di azienda}
\textbf{Complesso di beni organizzati dall'impreditore per l'esercizio dell'impresa.}
Strumenti attravero il quale l'impreditore svolge attività di impresa.


\textbf{È imprenditore chi esercita profesionalmente una attività economica organizzata al fine della
produzione e scambio di beni o servizi}.

\subsection{Caratteristiche impresa}
\begin{itemize}
	\item è attività: serie di atti diretti alla produzione o allo scambio di beni/servizi
	\item è attività economica: deve essere organizzata di modo che i costi siano inferiori ai ricavi
	\item è attività organizzata: per organizzazione si intende l'organizzazione di fattori produttivi(
		capitale, lavoro, macchinari, ecc)
	\item è attività professionale: esercizio svolto con continuità, determinazione intenzionalità
\end{itemize}

\subsection{Caratteristiche azienda}
\begin{itemize}
	\item apparato strutturale(locali, macchine, materie, ecc) di cui l'impresa si avvale.
\end{itemize}


\section{Definizione economico-aziendale di azienda}

L'azienda è:\begin{quotation}
    `` un istituto economico destinato a perdurare che,
per il soddisfacimento dei bisogni umani, ordina e svolge in
continua coordinazione la produzione o il procacciamento e il
consumo della ricchezza. ''

G. Zappa
\end{quotation}

\begin{quotation}
    `` un sistema di forze economiche che sviluppa,
    nell’ambiente di cui è parte complementare, un processo di
    produzione, o di consumo, o di produzione e di consumo
    insieme, a favore del soggetto economico, ed altresì degli
    individui che vi cooperano ''

A. Amaduzzi
\end{quotation}

\subsection{Caratteristiche di un azienda(economico-aziendale)}

\begin{itemize}
    \item organizzazione stabile, per coordinare e combinare il processo produttivo
    \item persone, che prestano le loro energie coordinandosi
    \item beni econimici, materiale, macchine, soldi, destinati ad essere utilizzati
    \item oggetto aziendale, attività svolta
    \item operazioni, sui beni svolto dalle persone per adempiere al fine dell'azienda
    \item fine(scopo)
    \item visione sistematica:
    i fatti aziendali non sono tra loro scollegati, tutto legato da causa effetto.
    \item autonomia:
    liberta di decisione (livello strategico, operativo)
    \item economicità: l'attività deve essere sempre ispirata alla logica:
    \begin{itemize}
        \item efficacia strategica: programmare e realizzare gli obiettivi
        \item efficacia operativa: realizzare la produzione a dovuti livelli qulitativi
    \end{itemize}
\end{itemize}


\section{Azienda come sistema aperto}

Per opravvivere l'azienda deve continuamente intrattenere continue relazioni di scambio con altre entità.


\section{Azienda come sistema socio-tecnico}

L’azienda è un sistema sociale all’interno del quale
operano risorse umane e tecniche (mezzi di produzione)
scarse (caratterizzate, cioè, da una disponibilità limitata),
organizzate e finalizzate al profitto.

\section{Impresa come sistema cognitivo}

La vera ricchezza dell'impresa è legata a:
\begin{itemize}
    \item immagine positiva
    \item avviamento sul mercato
    \item capacità di innovare
\end{itemize}


