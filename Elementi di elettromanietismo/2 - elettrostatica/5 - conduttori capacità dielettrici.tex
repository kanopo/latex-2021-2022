\section{Conduttori, capacità, dielettrici}
Il condensatore permette di immagazzinare energia e questa caratteristica
prende il nome di capacità.

\subsection{Capacità di un conduttore isolato}
La capacità elettrica solitamente è:
\begin{equation}
    Q=CV
\end{equation}

Esempio per una superficie sferica:
\begin{equation*}
    V = \frac{Q}{4\pi\epsilon_0R}
\end{equation*}
\begin{equation*}
    C = \frac{Q}{C} = 4\pi\epsilon_0R
\end{equation*}

\subsection{Induzzione totale}

Se sono presenti due conduttori in induzione totale(tutte le linne 
di forza del primo entrano nel secondo),
allora il secondo avrà cariche uguali con segno opposto.

\subsection{Condensatore}

La capacità edl condensatore è $C = \frac{Q}{V}[F]$ e si misura in Farad(1 F = 1C/V),

\subsection{Capacità di un condensatore piano}
Avendo un condensatore con area delle armature $A$ e distanza delle armature $d$,
\subsubsection{Densità superficiale}
\begin{equation}
    |\sigma| = \frac{Q}{A}
\end{equation}

\subsubsection{Campo elettrico tra le armature}
\begin{equation}
    E = \frac{|\sigma|}{\epsilon_0} = \frac{Q}{\epsilon_0 A}
\end{equation}

\subsubsection{Potenziale delle armature}
\begin{equation}
    V = Ed = \frac{Q}{\epsilon_0 A}d
\end{equation}


\subsubsection{Capacità condensatore}
\begin{equation}
    C = \frac{\epsilon_0A}{d}
\end{equation}

\subsection{Energia elettrostatica}
Energia potenziale di una carica in un campo elettrico:

Se ho un triangolo equilatero e ho delle cariche nei vertici, l'energia potenziale è:
\begin{equation*}
    U = \frac{1}{4\pi\epsilon_0}(\frac{q_1q_2}{d} + 
    \frac{q_2q_3}{d} + 
    \frac{q_1q_3}{d})
\end{equation*}
Quindi fa un ragioanmento individuale e poi si usa la sovrapposizione degli effetti.

Generalizzando:
\begin{equation}
    U = \frac{1}{4\pi\epsilon_0}
    \frac{1}{2}
    \sum_{\substack{i,j=1 \\ i\neq j}}^N
    {\frac{q_iq_j}{r_{ij}}}
\end{equation}

\subsection{Energia elettrostatica del condensatore}
Per sapere l'energia del condensatore durante il suo periodo di carica,
si usa un integrale.
\begin{equation}
    U = \int_0^Q{V'dQ' = \frac{1}{2}\frac{Q^2}{C}}
\end{equation}

\subsection{Densità di energia elettrostatica}
\begin{equation}
    u = \frac{U}{Ad}
    = \frac{1}{2}\epsilon_0E^2
\end{equation}

\subsection{Proprietà elettrostatica dei diellettrici}
La differenza di potenziale dipende anche dai materiali che si utilizzano come
dielettrico fra le armature.
La costante dielettrica relativa $\epsilon_r$ è il risultato di:
\begin{equation*}
    \epsilon_r = \frac{V_0}{V}
\end{equation*}
dove $V_0$ è la differenza di potenziale per dielettrico vuoto e $V$ 
con un dielettrico generico.

A volte si definisce \textbf{costante dielettrica} la quantità 
$\epsilon = \epsilon_r\epsilon_0$.
All'inserimento del dielettrico, il campo si riduce di un fattore 
$\frac{1}{\epsilon_r}$, la capacità aumento di un fattore
$\epsilon_r$ e l'energia potenziale cala di un fattore 
$\frac{1}{\epsilon_r}$.

Se invece si prova a cambiare il dielettrico con ancora il 
generatore collegato, quest'ultimo dovrà trasferire un pochino
di energia, quindi la carica del condensatore aumento di un fattore
$\epsilon_r$.