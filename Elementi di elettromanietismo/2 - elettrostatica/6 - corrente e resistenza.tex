\section{Corrente e resistenza}
\subsection{Intensità di corrente elettrica}
L'intensità di corrente elettrica è la variazione di cariche in un
lasso di tempo:
\begin{equation*}
    I = \frac{dQ}{dt}
\end{equation*}

\subsection{Velocità di deriva}
Il valore assoluto della carica $dQ$, che passa nella superficie $S$
nell'intervallo $dt$:
\begin{equation*}
    dQ = nSv_ddt|q|
\end{equation*}

dove:
\begin{itemize}
    \item $n$ = densità dei portaotri di carica
    \item $v_d$ = velocità di deriva
\end{itemize}

Per ottenerer la corrente, avendo trovato la variazione di carica, 
basta dividere per il tempo.

\subsection{Densità di corrente elettrica}
La densità di corrente, partendo dalla formula di prima della corrente,
dividiamo per la sezione e otteniamo:
\begin{equation*}
    J = nv_d|q|
\end{equation*}

\subsection{Consevazione della carica elettrica}
Da questa legge si capisce che se ho una carica uscente da una superficie,
la carica all'interno della superficie, è ridotta della quantità
uscità.
Questo ragionamento viene considerato quando un conduttore cambia di dimensioni
 provocando un cambio proporzioanle di densità della carica.

 Esempio:
 \begin{equation*}
     \oiint_{Schiuso}{\vec{J}d\vec{S}} =
     \iint_{S1}{\vec{J}d\vec{S}} + 
     \iint_{S2}{\vec{J}d\vec{S}} =
     - \iint_{S1}{\vec{J}d\vec{S_1}} + 
     \iint_{S2}{\vec{J}d\vec{S_2}} 
 \end{equation*}
 Serve a rappresentare la corrente entrante e uscente come densità per sezione
 così da poter fare una proporzione per l'esercizio.


 \subsection{Resistenza e legge di Ohm}
 \begin{equation*}
     R = \frac{V}{I}
 \end{equation*}

 \subsection{Resistività}
 \begin{equation*}
     R = \frac{\rho l}{S}
 \end{equation*}

 Dove $\rho$ indica la tipologia di materiale, l indica la lunghezza e S è la superficie.

 \subsection{Legge di Ohm in funzione di densità di corrente e campo}
\begin{equation*}
    \vec{J} = \sigma \vec{E}
\end{equation*}

Un campo elettrico E produce una denstià di corrente j che dipende dalla 
conducibilità del materiale (sigma).

Riassuntone:
\begin{itemize}
    \item $\vec{J} = \sigma\vec{E}$, legge di ohm in un punto interno del materiale
    \item $V = IR$ legge di ohm in un conduttore
    \item $\rho = \frac{RS}{l}$
    \item $\sigma = \frac{l}{RS}$
\end{itemize}
Rho e sigma sono opposti.

\subsection{Semiconduttori}
Materiali che si comportano comeisolanti in certe condizioni e da conduttori
in altre condizioni.
